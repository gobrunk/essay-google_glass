\documentclass[a4paper,12pt]{article}
\usepackage[utf8x]{inputenc}
%\usepackage[latin1]{inputenc}
\usepackage[T1]{fontenc}
\usepackage[francais]{babel}
\usepackage{lmodern} % Pour changer le pack de police
\usepackage[top=2cm, bottom=2cm, left=2cm, right=2cm]{geometry}
\usepackage{setspace} % Interligne de 1,5
\onehalfspacing
\usepackage{parskip}
\setlength{\parindent}{4em} % Alinéa de 4 espaces
\setlength{\parskip}{2em} % Interligne de 2 entre chaque paragraphes
\usepackage{fancyhdr}
\pagestyle{fancy}
\fancyhf{}
\lhead{Rodolphe \textsc{Guillaume}, François \textsc{Kremer}}
\rhead{\today}

\begin{document}

\begin{center}
\huge
Google Glass
\end{center}

The Google Glass, the smart pair of glasses from Google, was first released in 2013 at \$1500 but it ended up being a massive financial failure. Last year, Google decided to stop the production and the sale of the glasses. However, in 2016 Google is launching the new foldable glasses for the workplace, the Google Glass Enterprise Edition. These have a brand new software and hardware designed especially for work purposes. This suggests that the Google Glass will be a must in the workplace in the coming decades, like computers are nowadays.

Idea 1.

Idea 2.

Idea 3.

Et plus si tu veux.

Twenty years ago only a few employees had a personal computer at their desk office. Several years later, most of the office employees use a computer in order to do their work more efficiently. Beside, smartphones enabled workers to make calls and send mails while being away from their computers. It increased their efficiency once again. Google Glass is the next step, allowing workers to use their voice, their hearing and their sight in the dozens of applications of the glasses, on top of its AI which makes it the most powerful intelligent personal assistant.

\end{document}

